\chapter{Future Work and Conclusion}\label{conclusion}

\section{Future Work}
This thesis has laid the groundwork for a novel approach to evaluating Large Language Models (LLMs) in the domain of long-form question answering (LFQA), specifically in the medical domain. However, there is substantial room for further research and development. Future work should focus on:

\begin{enumerate}
    \item \textbf{Dataset Expansion:} Extending the dataset to include more diverse queries and documents, particularly in languages other than English, and other domains beyond medical queries, to enhance the generalizability of the findings.
    \item \textbf{Algorithmic Improvements:} Developing more sophisticated methods for preprocessing web documents and extracting their core messages, which could lead to more challenging and representative datasets for LFQA.
    \item \textbf{Multidimensional Evaluation:} Refining the evaluation metrics to better capture the multidimensional nature of answer quality, including aspects like the factual accuracy, coherence, and user engagement.
    \item \textbf{Cross-Model and Cross-Lingual Comparisons:} Conducting comparative analyses across different LLMs and languages to understand the nuances in model performances and linguistic variations.
    \item \textbf{User Experience Studies:} Investigating how users interact with LLMs in real-world scenarios, especially in the medical domain, to understand user expectations and satisfaction.
    \item \textbf{Ethical and Privacy Considerations:} Addressing ethical and privacy concerns related to the use of LLMs for medical advice, ensuring that such systems are safe, reliable, and respectful of user privacy.
    \item \textbf{Integration with Healthcare Systems:} Exploring the potential integration of LLMs into existing healthcare systems, assessing the practical challenges and benefits of such integration.
\end{enumerate}

\section{Conclusion}
In conclusion, this thesis has introduced a new approach to evaluating LLMs for LFQA tasks using a retrieval-based implicit evaluation method. Our findings, as discussed in Chapter \ref{chapter:results} and Chapter \ref{discussion}, highlight the multidimensional nature of answer quality in LFQA and the various factors that influence the effectiveness of LLMs in this context.

The proposed method has shown promise in reducing the reliance on human evaluations, thereby potentially lowering the costs and increasing the consistency in evaluating LLMs. However, as outlined in Chapter \ref{sec:scope-and-limitations}, this approach is not without limitations and should be considered a starting point for future research.

The rapid development and widespread adoption of LLMs, especially in applications like ChatGPT, as mentioned in Chapter \ref{structure}, underscore the importance of robust and scalable evaluation methods. The healthcare domain, in particular, presents unique challenges and opportunities for the application of LLMs. It is crucial that we continue to refine our evaluation methodologies to ensure that these powerful tools are used effectively and responsibly.

This thesis represents a step forward in the ongoing journey to understand and harness the potential of LLMs in answering complex, open-ended questions, particularly in sensitive domains like healthcare. The path ahead is filled with exciting possibilities and challenges, and it is hoped that this work will inspire and inform future endeavors in this dynamic field.

