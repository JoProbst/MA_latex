\chapter{Experimental Setup}
This chapter describes the experimental setup of this thesis, including the data collection and preparation, the retrieval pipeline development and the generation of LLM responses.

\section{Overview}\label{sec:overview}
The final goal of this project is to have a pipeline with which the answers of different LLMs can be ranked, in comparison to human answers.
This means, we first need a dataset, consisting of questions as well as human answers to these questions.
Those human generated answers need to be ranked according to how well they answer the given question.
Then, we need to develop a retrieval pipeline, which is able to retrieve the best answer to a given question in accordance to the ranking of the human answers.
Finally, the answers generated by different LLMs can be ranked using the developed retrieval pipeline.
\\\\
This allows us to compare long form question answering capabilities between different LLMs or between different prompting strategies for the same LLM.
Additionally, an expected range of performance for a given LLM can be established, given that the answer of a LLM differs every time it is prompted with the same question.
\\\\
To be able to do this, we first need to collect a suitable dataset, since a benchmark like this has not been formulated before.
Then, different retrieval methods have to be compared on the dataset, to find the best performing one.
Finally, the LLMs have to be selected and the answers generated by them have to be ranked using the best retrieval method.
\\
This process is described in detail in the following sections.

\section{Data Collection and Preparation}
The used dataset is based on the CLEF eHealth 2021 Task 1 dataset \cite{goeuriot:2021}.
It was originally intended to evaluate the ability of retrieval systems to provide credible, readable and relevant answers to laypersons' health questions.

\subsection{CLEF eHealth 2021 Task 1 Dataset}
The dataset consists of 55 health related queries, which either stem from Reddit or Google search trends.
While the Reddit-sourced queries are well-formulated questions about specific health topics, the queries from Google search trends are not necessarily phrased as questions but rather as classical search queries.
\\
In addition to the queries, the dataset includes a collection of web documents and social media content.
The web documents were mainly obtained from the CommonCrawl archive, encompassing a diverse range of 600 domains.
This list of domains was created by executing medical queries via Microsoft Bing API and was augmented by adding known reliable and unreliable health-related websites.
The dataset was expanded by incorporating social media comments and posts from Reddit and Twitter.
These were collected by manually generating search queries based on 150 pre-selected health topics and retrieving relevant posts and comments from these platforms.
\\
To get evaluations in each of the three categories (credible, readable and relevant), each query was assigned 250 documents based on rank-biased precision (RBP)-based Method A~(\cite{moffat:2008}).
RBPA is a method for choosing which documents from a pool of documents to select for evaluation by human annotators.
The pool of documents in this case is the collection of web documents and social media content returned by the participating teams of the shared task, as well as the organizer's baseline systems.
From this, documents are evaluated based on their rank in different runs, and then scored according to different metrics.
A more detailed description of the pooling method can be found in \cite{lipani:2017}.
\\
After the 250 documents per query are selected, the documents are evaluated by humans for credibility, readability and relevance.
In total, there are 26 annotators, each annotating documents for between 1 and 4 queries completely.
The annotators were not medical experts, but received written training material.
In the end, annotations were made for  a total of $11 357$ unique documents.
This differs from the total amount of annotations which is $12 500$, since some documents were annotated multiple times but for different queries.
\\
The annotations for relevance and readability are in the categories 0 (not relevant/readable), 1 (somewhat relevant/readable) and 2 (highly relevant/credible).
For scoring credibility, a category 3 exists, which is also interpreted as highly credible.
A reasoning for this is not given in the original paper.
\\
The process of collecting and adapting the dataset is described in the next section.

\subsection{Dataset Collection}

\subsection{Preprocessing}

\subsection{Dataset Statistics}


\section{Retrieval Pipeline Development}

\subsection{Baseline Models}

\subsection{Transformer Models}

\section{Generation of LLM Responses}


\subsection{Selection of Language Models}

\subsection{Prompting approaches}
